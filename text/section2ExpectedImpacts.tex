% !TEX root = ../proposal.tex

\subsection{Expected impacts}
\label{sec:impact_expected}
\subsubsection{Expected Impacts related to the Work Programme, SRA, and MAR} 
%   * improving innovation capacity and the integration of new knowledge
%     (strengthening the competitiveness and growth of companies by
%     developing innovations meeting the needs of European and global
%     markets; and, where relevant, by delivering such innovations to the
%     markets;
In the following, we list the specific expected impacts from the Work Programme, the Strategic Research Agenda (SRA), and the Multi-Annual Roadmap (MAR) and describe how \Project{} contributes towards them.

\newcounter{ImpactCounter}
\textbf{Expected Impact \stepcounterarabic{ImpactCounter} : \textit{``Deploy robotics technologies in new application domains''}} (Work Programme) \\
\textbf{\textit{``In the short to medium term robotics technologies will provide opportunity for product improvement and safety [...]. Opportunities arise around navigation for tasks such as navigation support at night, in GNSS denied zones, and for disabled and ageing drivers. Partial or fully autonomous driving will be used in low speed urban transport [...].''}} (MAR, p.109) \\
The market share of automated transportation systems is currently minuscule and primarily dominated by private industrial settings. The successful demonstration of the technology developed during the course of \Project{} will pave the way for accelerated market take-up of targeted Robotic urban transportation services -- thereby initiating the penetration of autonomous robots on public roads in more general. This is in particular the case for the valet parking in urban environments applications developed within this project. The technologies developed in this project will not only serve this application, however, but autonomous driving in city environment in general. These technological advances are manifold and stepwise. The improvements in local environment perception, sensor fusion, semantic classification and especially the aggregation and fusion of this information from multiple vehicles in a cloud-based server backend will break new grounds in urban street safety. Complex situations can be assessed on the basis of information that is both spatially and temporally beyond the range of a single vehicles sensors suite. These novel approaches will gradually be integrated into advanced driver assistance systems and continuously reduce the risk of accidents and the number of fatalities. The efforts towards a first large-scale lifelong localization and mapping system will offer additional benefits, such as accurate navigation aid in GPS denied environments and accurate accident reconstruction.


\textbf{Expected Impact \stepcounterarabic{ImpactCounter} : \textit{``Improve Technology Readiness Levels of robotics technologies.''}} (Work Programme) \\
The proposed system addresses the tremendous challenges associated with autonomous transportation in open, unstructured and populated environments. This is only achievable by jointly improving the associated key Robotic base technologies in tandem -- including local environment perception, accurate lifelong localization and mapping, scene understanding and the aggregation and management of this information from multiple vehicles in a cloud-based server backend. 
 %To this end local real-time sensor perception will be combined and enhanced with long-term semantic data from the mapping backend, enabling a complete and profound assessment of the vehicles' environment via scene understanding modules. This will form the basis upon which appropriate action by the decision making and control module can be taken.
For all of these technologies substantial increase in technology readiness level, generally to TRL 6 by project end, is to be expected -- see Section~\ref{sec:trl} for specifics. 


\textbf{Expected Impact \stepcounterarabic{ImpactCounter} : \textit{``Increase Europe's market share in industrial robotics to one third of the market and maintain and strengthen Europe's market share in professional (50\%) and domestic (20\%) service robotics by 2020.''}} (Work Programme) \\
Several traditional industries are at a crossroad where robotic technology are expected to disrupt conventional business models and result in new and innovative branches of industries. \Project{} brings together a leading player in the Automotive industry (\VW) with specialists in Cloud Computing and Data Analytics (\IBM). Together with the strong expertise of the academic partners on key basic Robotic technologies and the proposed research and development steps within this project results in a path for immediate exploitation of the project results and strengthening of Europe's market share in this growing field.
Advances developed in the project will be targeted for exploitation and commercialization by transferring the results in new product lines as well as start-ups originating from the academic partners (as detailed in section~\ref{sec:expl}, on page~\pageref{sec:expl}).


\textbf{Expected Impact \stepcounterarabic{ImpactCounter} : \textit{``Contribute to an inclusive society through robotic technologies''}} (Work Programme) \\
The project's approach towards rapid market take-up of a limited but targeted citizen-centric high-value automated driving applications strongly contributes towards a more inclusive society. The developed valet parking demonstrator will strongly improve the individual mobility for elderly and citizens with physical handicaps in particular, allowing them to participate more and longer in social activities, thus mitigating potential isolation. In a similar vein, they would result in an inclusive effect on the group of citizens struggling with affording transportation in general and a personal vehicle in particular.


\textbf{Expected Impact \stepcounterarabic{ImpactCounter} : \textit{``Increase Industry-Academia cross-fertilization and tighter connection between industrial needs and academic research via technology transfer, common projects, scientific progress on industry-driven challenges.''}} (Work Programme)\\
Members of the \Project{} consortium come from both leading industry and academic environments and hence collaboration and direct transfer of knowledge is key to the success of the project. The industrial partners with their innate knowledge of end customers needs will be involved in defining product requirements. They will lead the definition of the design specifications and the university partners will conduct world class research enabling the development of the required technology modules. Finally, the communication protocols and management strategies developed and used throughout \Project{} will be exemplary for conducting industry-academia cross fertilization and foster a tighter connection between all parties involved.


\textbf{Expected Impact \stepcounterarabic{ImpactCounter} : \textit{``Ensure sufficient numbers of well-trained professionals required by the growth of the industry.''}} (Work Programme) \\
All involved partners are highly committed to education and knowledge transfer. Within this project, the academic partner will employ and educate new PhD students. These students will benefit from the unique opportunity to train and shape their skills on an ambitious project with a clear focus on market-driven products. The industry partners will train their employees and offer them a cutting edge further education within this project. The interaction between the partners, both academic and industrial, will further improve and upvalue the education through cooperation, mutual exchange of ideas and transfer of knowledge. All in all, the educational efforts made in this project guarantee a high-class training of the next generation engineers and researchers.


\textbf{Expected Impact \stepcounterarabic{ImpactCounter} : \textit{``Ensure wide use of shared resources.''}} 
(Work Programme) \\
The consortium intends to develop the system using established open source middle ware for communication, as well as a single project-wide development, testing and deployment repository to establish the basic requirements enabling wide use of shared resources. All of the academic consortium members have a history of releasing core technology as open source software. Together, these items ensure high impact of the project in terms of reuse of results in other projects, and by other researchers. Similarly, the consortium has identified synergies with other previous and currently running projects (See Section~\ref{sec:ambition}) and these will be exploited to their maximum potential. Substantial R\&D know-how, software and the initial vehicle platforms will be carried over and built upon from the V-Charge project in particular.




\textbf{Steps to achieve the desired impacts:}\\
The \Project{} consortium will take the following measures in order to achieve the expected impacts:
\begin{denseItemize}
\item Industry-led definition of product specifications ensuring alignment of the developed technology to the needs of the European and international markets.

\item Collaborate between partners to develop innovative product and technology designs which can be seamlessly transfered to real world scenarios, and facilitate interoperability with other products on the market by utilizing standardized communication protocols and representation frameworks. This will ensure minimal barriers to entry when the technology is ready to be put on the market.

\item Repeatedly evaluate components and the overall system according to the spiral life-cycle approach described in Section~\ref{sec:workoverview} leading to a robust system, capable for long-term operation, and easy to use with a short setup time.

\item Disseminate the results, open datasets, and open-source software to the international public, including research bodies, students and general public, but also potential customers of the navigation components as well as the final system (see Section~\ref{sec:diss}). 

\item Communicate the results through videos, press releases, and demonstrations to raise awareness of the project, its goals, its scientific results, and the targeted market applications (see Section~\ref{sec:communicationact}). 
\end{denseItemize}
 
