% !TEX root = ../proposal.tex

\subsubsection{Risk Assessment, Reviewing and Measures for Success}
It is impossible for an ambitious project like \Project{} to steer clear of all risks. A key choice towards reducing risks is that we exploit the iterative spiral development model so that individual research and developments steps remain short and system-wide integration and evaluation starts early and is revisited multiple times during the project. This is particularly important for \Project{} since it depends heavily on the integration of a broad set of components at the edge of the state of the art. 

Consequently, in \Project{} all partners will contribute to integration early, even while the contributing WPs are only partially completed. This way, integration
difficulties, if any, will occur early in the lifetime of the project and, hence, there will be a sufficient amount of time to resolve them. Additionally, the prior experience and mutual trust of the consortium partners, ensures that risks will be identified, discussed and tackled immediately.

We have assessed the possible risks for every work package and developed suitable contingency plans. The individual risks and associated contingency plans are detailed in the following sections. The risks related to project management, dissemination and exploitation are described in Section~\ref{sec:management}. During the project, we continue to monitor these and emerging other risk factors closely.

