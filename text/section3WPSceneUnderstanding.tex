% !TEX root = ../proposal.tex
%comment

\paragraph{\WPSceneUnderstanding: \WPSceneUnderstandingTitle\\}
%%%%%%%%%%%%%%%%%%%%%%%%%%%%%%%%%%%%%%%%%%%%%%%%%%%%%%%%%%%%%%%%%%%%%%%%%%
%% WP
%%
{\noindent\wptablefont
\label{wp7}

\wptableheaderA{\WPSceneUnderstandingTitle}{\WPSceneUnderstanding}{M3}{RTD}{\PRAGUE}
\wptableheaderB{\WPSceneUnderstandingVW}{\WPSceneUnderstandingETHZ}{\WPSceneUnderstandingIBM}{\WPSceneUnderstandingCLUJ}{\WPSceneUnderstandingPRAGUE}

\headerBox{Objectives}{}

This work will target all aspects contributing to understanding the local scene in the surroundings of the car, including driving mode and behavior of other roads occupants. To this end, it interfaces to the multi-sensorial fused perception data from \WPPerception as well as the server-side lifelong map from \WPMapping. The aim of this work package is to provide a local scene analysis in terms of situation and expected behavior of the other traffic occupants, including semantic annotations related to the obstacles as well as the prediction for object maneuvers. Split into individual items, the key objectives include:
\begin{denseItemize}
\item Development of on-board high-level\,/\,scenario-based scene representation with optional support from server side data
\item Development of context based surrounding road users' behavior analysis
\item Achieving ego driving behavior understanding
\end{denseItemize}



\begin{tasks}{\WPSceneUnderstandingNo}

\item  {\bf Long-term semantic scene understanding}
  \taskpartners{\IBM}{\ETHZ, \VW}
\label{task:sceneunderstanding:semanticunderstanding}

This task will employ the knowledge accumulated and maintained by the perception and lifelong mapping modules over extended periods of time to derive a high-level assessment of semantic information in the scene based on experience. This on one hand involves scene segmentation as well as distinction between static and dynamic objects to further improve the online perception routines in \WPPerception as well as localization performance in \WPMapping. On the other hand it involves the derivation of insights based on a statistical analysis of the semantic information in the lifelong map of \WPMapping.


\item  {\bf Scenario based scene understanding}
  \taskpartners{\PRAGUE}{\VW, \CLUJ}

This task will define and develop a structure to effectively represent the environment surrounding the vehicle in terms of \emph{scenarios} (instead of the \emph{structures and objects} tackled in \WPSceneUnderstandingNo.\ref{task:sceneunderstanding:semanticunderstanding}). Scenarios will encode various information, including the driving mode currently adopted on the vehicle (e.g., urban, low speed, parking, etc.) coupled with traffic conditions and road user behavior. For road user behavior understanding, dynamic object tracking techniques developed in \WPPerception are extended by complementing them with visual perception information and high level reasoning, to create a profile of each road user surrounding the ego vehicle. The task involves the development of Machine Learning algorithms to predict and classify specific behavior such as: willingness to overtake; changing lanes; presence of a traffic jam; road works, and slopes. The task further involves the modeling and analysis of relations between the current context and detected users' behavior, including behavior at rush time; reaction to traffic jams; reaction to overtaking; and weather related behavior.

\item  {\bf Scene prediction}
  \taskpartners{\PRAGUE}{\VW}

The goal of this task is to provide a prediction of the scene evolution in the near future (~10s). This prediction includes the movement of other traffic participants and is based on the current high-level scenario based understanding of the scene as well as local perception data and offline semantic maps. Proper handling of the uncertainty of the interpretation will be a central issue.

\item  {\bf Self-assessment}
  \taskpartners{\PRAGUE}{\VW}

This task will focus on the development of experimental methods for self-profiling. We expect the system to be able to self-assess its performance in relation with the current scenario and context information generated by the previous tasks, learning not only to recognize scenarios, but also which is the best action to be taken in a given context -- thereby eventually allowing it to bypass some elements of the navigation stack or imposing some constraints upon it (e.g. reduce speed). This task involves general data integrity assessment for which relevant inputs are map and localization accuracy (\WPMapping), map integrity verification (\WPPerception), estimation of field of view - including view obstructions (\WPPerception). The result of this comparison will be passed on to the Tactical Planning (Task~\WPNavigation.\ref{task:wpnav:tact}) as well as Mission Executive (Task~\WPNavigation.\ref{task:wpnav:mission})

\end{tasks}


\begin{deliverables}{\WPSceneUnderstandingNo}

%new deliverable structure for tech WPs as suggested by Wojtek:
\item {\bf Software specification and architecture for scene understanding} \putright{\bf M6}
   \delresponsible{\PRAGUE, all other partners}

In this report detailed input and outputs of scene understanding will be produced. Input will include the specific interfaces from other Work Packages. Outputs will define the format of the data produced to detect the current scene and predict the expected behavior of traffic occupants. Also the interfaces between the individual modules involved into scene understanding will be provided.

\item {\bf First development and integration cycle of scene understanding} \putright{\bf M20}
   \delresponsible{\PRAGUE, all other partners}
	
The deliverable covers the first development and integration cycle and reports on current state of all tasks within the Work Package.
This report details statistical insights on semantic aspects of urban scenes (\eg parking lot usage). It will furthermore consist of software modules able to extract said information from the map representation of \WPMapping. 
It will also provide an exhaustive list of urban driving scenario and their characteristics. It will furthermore consist of software modules able to profile the road user behavior in terms of dynamic and predicted intentions. 

\item {\bf Second development and integration cycle of scene understanding} \putright{\bf M44}
   \delresponsible{\PRAGUE, all other partners}
	
The deliverable covers the second development and integration cycle and reports on current state of all tasks within the Work Package..
It will report details statistical insights on semantic aspects of urban scenes (\eg parking lot usage). It will furthermore consist of software modules able to extract said information from the map representation of \WPMapping. It will also provide a report and a software able to analyze the ego-behavior and synthesize it in terms of high level representation.

\end{deliverables}

%\clearpage

\mosriskheader

%------------------------------------------------------------------------
\begin{SuccessTable}{Task}{Measures for Success}
  Task~\WPSceneUnderstandingNo.1: Long-term semantic scene understanding & The software module is capable of correctly analyzing statistical properties of relevant semantic elements of the environment (\eg parking lots usage). \\ \hline
  Task~\WPSceneUnderstandingNo.2:  Scenario based scene understanding & Software module integrated into the system and able to classify the scenario in accordance with the criteria and with the performance identified in the specification phase\\ \hline
  Task~\WPSceneUnderstandingNo.3: Scene prediction & Object motion predictions guided by scene understanding. Correct and stable predictions for time horizon up to 2-10 seconds (depending on perception horizon)\\ \hline
  Task~\WPSceneUnderstandingNo.4: Self-assessment & Software module integrated into the system and able to assess system limits in accordance with the criteria identified in the specification phase
\end{SuccessTable}

\vspace{1cm}

%------------------------------------------------------------------------
\begin{RiskTable}{\WPSceneUnderstanding-specific Risks}{Contingency Plans}

High-level road user as well as scenario analysis might be cultural and country dependent, making it difficult to create universally valid user behavior prediction techniques. & medium & The proposed system will be developed for the European context, while keeping the architecture generic. This allows it to be extended seamlessly by feeding the machine learning algorithms with new datasets. \\ \hline
Bottle-necks in the communication and cloud reasoning channel could influence driving performance & low & The vehicle needs to be safe at all times based on vehicle-side perception. If safety based on cloud services cannot be guaranteed, the vehicle-side navigation elements need to limit maximal velocity. 
\end{RiskTable}
}

