% !TEX root = ../proposal.tex

\subsection{Objectives}
\label{sec:objectives}

\Project{} aims to address the outlined transport-related challenges by providing key contributions that will enable gradual automation of and collaboration among vehicles -- and as a result facilitate a safer, more inclusive and more affordable transportation system. While fully automated transportation is not expected to be achieved in the short term, the Multi Annual Roadmap encourages a two-pronged approach with an aim of

\begin{denseItemize}
	\item focusing on specific high-value market domains where an expedited market take-up appears achievable, and 
	\item maturing associated base technologies towards higher Technology Readiness Levels (TRL).
\end{denseItemize}
	
%wde: paragraph to complex.
%wde: do not like the overcome-circumvent sentence. if offers no new content and is not explained later 	
%The \Project{} consortium fully shares and subscribes to this strategy and aims to overcome the presented hurdles where necessary and circumvent them where feasible. To this end the project aims at an integrated, reinforcing package composed of the following targets.

%\textbf{Exploiting existing instrumented vehicle infrastructure:} in aggregate, presently deployed instrumented cars represent the most advanced, massive and pervasive sensing, communication, storage and computation platforms in operation today. Being typically parked for 22 hours or more per day, they represent a largely paid for resource waiting for exploitation and monetization. Our aim is to tap into this resource as a basis for a bootstrapped, integrated approach towards automated transportation and other services, where instrumented (and in the future automated) cars serve as distributed sensing, storage and compute platforms providing the massive amounts of real-time information necessary to further decisively advance the TRL of perception, cognition and motion ability -- and, hence ultimately safety.

%\textbf{Targeting specific high-value applications serving particular citizen groups:} a strict focus on urban neighborhood areas enables \Project{} to obtain the maximal impact per citizen. Furthermore, urban areas are most conductive for an accelerated deployment of safe automated transportation services due to the frequent presence of road-side parking (and the associated perception infrastructure enabled by parked cars, as described above), and reduced speed areas (up to 30\,km/h). Our aim is to develop and under these conditions exploit a set of temporally staggered, targeted, high-value applications that include valet parking in open neighborhood areas (i.e., automated pick up and drop off at home) and delivery of goods on the last mile. The exploitation of these application scenarios would represent a major alleviation for elderly as well as for people with physical handicaps. They would further result in an inclusive effect on citizens struggling with affording transportation in general and a personal vehicle in particular.

%\textbf{Developing step advancements in key base-technological areas:} from a technology perspective, several base technological competencies become critical for enabling and powering the above described automated urban outdoor applications and services. The aim of \Project{} is to develop and extend these competences in tandem that include accurate metric localization and distributed geometrically consistent mapping in large-scale, semi-structured areas; representations and mechanisms for efficient and cost-effective long-term data management across devices; and scene understanding, in particular learning and classification of objects in semi-structured outdoor environments and their association with web-based knowledge.

%Via the presented set of mechanism the project aims to achieve a mutually reinforcing cycle of research, development and accelerated deployment of increasingly automated yet targeted services and mobility solutions in urban areas -- for the good of the citizens living there.

%The \Project{} consortium fully shares and subscribes to this strategy and thus proposes the following approach:
The consortium fully shares and subscribes to this strategy and established the following objectives to be reached within \Project{}:

\textbf{Exploitation of close to market technology.}
\Project{} is based on a realistic, close-to-market vehicle platform. A combination of \emph{stock} or \emph{pre-development} sensors such as cameras, radars and laser scanners will be employed. On the one hand, the choice of sensor configuration will be guided by the requirement to cover all relevant areas of interest and on the other hand by keeping resemblance to pre-development sensor configurations and keeping the overall number of sensors limited. \Project will also investigate exploitation of fleet data of connected vehicles. Focus will be placed on aggregating and sharing \emph{experience} about the static parts of the environment and typical motion/behavior patterns rather than the \emph{current snapshot} of the dynamics of a scene. This lets the system fully adapt to the communication link available as opposed to requiring guarantees for some specific performance qualities. To this end it is also irrelevant if the cars of the fleet are driven manually or are automated.


\textbf{Pushing forward key technologies.}
The challenges of bringing automated driving to urban areas originate from the complexity of urban environments and more importantly -- urban traffic, including more complex road geometry, varying or missing lane boundaries, more diverse road user types, significantly more crowded and cluttered scenes and more complex interactions between traffic participants.

In order to adequately address this complexity \Project will focus on advancing the following technologies:
\begin{denseItemize}
	\item robust, general 360\degree{} object detection and tracking employing low-level spatio-temporal association, tracking and fusion mechanisms 
	\item accurate metric localization and distributed geometrically consistent mapping in large-scale, semi-structured areas
	\item representations and mechanisms for efficient and cost-effective long-term data management across devices
	\item scene understanding, starting from detection of semantic features, classification of objects, towards behavior analysis and intent prediction
\end{denseItemize}

As a result of this strategy \Project expects a significant technological progress that will benefit \emph{all} levels of automation: from driver assistance in the short-term to full automation in longer term -- across a broad range of applications.


%To bring all of these aspects together, our main objective is to build up a demonstrator platform consisting of two automated cars and a cloud environment to demonstrate automated transportation at low speeds in urban environments. To kick-start development and build on existing know-how we build on the outcome of the V-Charge project, and will reuse the project cars while the new vehicle is under development. This is described in more detail in Section 1.3. 

In order to bring the above aspects together, the main goal of \Project consortium is to build up a demonstrator platform consisting of two automated cars and a cloud environment to demonstrate automated transportation in urban environments. To kick-start development and build on existing know-how we build on the outcome of the V-Charge project, and will reuse the project cars while the new vehicle is under development. The specific concept and approach is described in more detail in Section \ref{sec:concept}. 

