% !TEX root = ../proposal_TA_Sec_4_5.tex

% This section is not covered by the page limit.
% The information provided here will be used to judge the operational
% capacity.
\clearpage
\section{Members of the consortium}


\subsection{Participants (applicants)}
\label{sec:participants}
% Please provide, for each participant, the following (if available): 
% - a description of the legal entity and its main tasks, with an
%   explanation of how its profile matches the tasks in the proposal;
% - a  curriculum vitae or description of the profile of the persons,
%   including their gender, who will be primarily responsible for
%   carrying out the proposed research and/or innovation activities;
% - a  list of up to 5 relevant publications, and/or products, services
%   (including widely-used datasets or software), or other achievements
%   relevant to the  call content; 
% - a  list of up to 5 relevant previous projects or activities,
%   connected to the subject of this proposal;
% - a  description of any significant infrastructure and/or any major
%   items of technical equipment, relevant to the proposed work;
% - [any other supporting documents specified in the work programme for
%   this call.]
{\small
\begin{tabular}{|l|p{7.6cm}|p{2cm}|p{3.0cm}|}
\hline
\highlightCell  Participant No &\highlightCell  Participant organization name & \highlightCell Part. abbrev. &\highlightCell Country
\\ \hline \hline
  \VWNo~(Coord.) & Volkswagen AG & \VW & Germany
  \\ \hline
  \ETHZNo  & Eidgen\"{o}ssische Technische Hochschule Z\"urich & \ETHZ & Switzerland
  \\ \hline
  \IBMNo  & IBM Research GmbH & \IBM & Switzerland
  \\ \hline
  \CLUJNo  & Universitatea Tehnic\u{a} din Cluj-Napoca & \CLUJ & Romania
  \\ \hline
  \PRAGUENo & \v{C}esk\'{e} Vysok\'{e} U\v{c}en\'{i} Technick\'{e} v Praze & \PRAGUE & Czech Republic
  \\ \hline
\end{tabular}
}

%\clearpage
\subsubsection{\VW}

\newHeaderBox{Volkswagen Group Research}{\bf\VW}

The Volkswagen Group, headquartered in Wolfsburg, Germany, is one of the world's leading automobile manufacturers and the largest carmaker in Europe. Volkswagen operates 118 production facilities in 31 countries and has a broad product range stretching from ecological passenger cars to luxury and sports cars, light \& heavy duty trucks and commercial vehicles. In 2014, Volkswagen employed nearly 593,000 people across the group and delivered 10,137 million vehicles (12.9\% of the world market) to customers. Annual turnover exceeded \euro{}202 billion with expenditures for R\&D of \euro{}11.5 billion. 45,742 employees within R\&D account for about 7.7\% of the total workforce. Volkswagen has extensive experience in the co-ordination of and participation in national and EC funded projects and networks involving European car manufacturers, suppliers, research organisations and universities.

Volkswagen Group Research has been very active in the fields of Advanced Driver Assistance Systems and Automated Driving for at least last decade. It has taken part in or lead numerous EC funded projects in the field, including: V-Charge, AdaptIVe, interactIVe, ARTRAC, HAVEit, euroFOT, INTERSAFE-2. This, as well as high internal investment into research activities has lead to extensive experience in such key areas as perception, scene interpretation, decision making, navigation, motion planning and - last but not least - vehicle setup and system integration.

%DELIVER, ELVA, SmartBatt, ALIVE, OVERSEE, PreDriveC2X, DACOTA, FIMCAR, MID-MOD.

\begin{keypubs}{\VW}
	\item
	Topfer, D.; Spehr, J.; Effertz, J.; Stiller, C., "Efficient Road Scene Understanding for Intelligent Vehicles Using Compositional Hierarchical Models," Intelligent Transportation Systems, IEEE Transactions on , vol.16, no.1, pp.441,451, Feb. 2015
	\item
	Muehlfellner, P.; Furgale, P.; Derendarz, W.; Philippsen, R., "Evaluation of fisheye-camera based visual multi-session localization in a real-world scenario," Intelligent Vehicles Symposium (IV), 2013 IEEE , vol., no., pp.57,62, 23-26 June 2013
	\item
	Aue, J.; Schmid, M.R.; Graf, T.; Effertz, J., "Improved object tracking from detailed shape estimation using object local grid maps with stereo," Intelligent Transportation Systems - (ITSC), 2013 16th International IEEE Conference on , vol., no., pp.330,335, 6-9 Oct. 2013
	\item
	Knaup, J.; Homeier, K., "RoadGraph - Graph based environmental modelling and function independent situation analysis for driver assistance systems," Intelligent Transportation Systems (ITSC), 2010 13th International IEEE Conference on , vol., no., pp.428,432, 19-22 Sept. 2010
	
\end{keypubs}


\newHeaderBox{Principal Investigators}{\bf\VW}

{\bf Dr. Lutz Junge} received his doctoral degree in physics with focus on nonlinear system dynamics at the university of G\"ottingen in 2000. He joined VW in 2008 and was the coordinator of the national funded project FAMOS and lead several VW internal projects on automated driving.  
Now he leads the department Driver Assistance and Automated Parking at Volkswagen Group Research

{\bf Dr.-Ing. Michael Darms} received his doctoral degree on Sensor Data Fusion in 2007 at Darmstadt University of Technology, Institute of Automotive Engineering. From 2002 to 2006 he worked as scientific assistant at this institute. 2006 to 2007 he worked as visiting researcher at Carnegie Mellon University, Robotics Institute. 2008 to 2010 he worked for Continental AG - Advanced Engineering, 
leading the Perception Group. From 2010 to 2014 he was Technical Project Manager for Camera Systems
at Continental, Division Chassis\&Safety. He joined VW Group Research in 2014 and is now leading the department Sensors and Fusion. 

{\bf Wojciech Derendarz} received his MSc in Telecommunications and Computer Science from the Technical University of Lodz, Poland in 2006. He was with University of Braunschweig in 2007, taking part in the finals of the Darpa Urban Challenge and joined VW in 2008. Since then he has worked in the field of image processing and since 2011 is the VW project leader for the EC funded project V-Charge responsible for overall system integration.

{\bf Kai Homeier} received his diploma in computer science at the Braunschweig University of Technology in 2006, focusing on robotics. After that he was responsible for path-planning at Team Carolo, the Darpa Urban Challenge finalist from Braunschweig University. From 2008 he worked for VW Group Research on the topic of graph based environment modelling. He joined VW in 2011 working on path-planning for automated driving, as shown on CES 2015.

{\bf Sebastian Grysczyk} received his MSc in Computer Vision \& Computational Intelligence from the South Westphalia University of Applied Sciences in 2012. He joined Volkswagen Group Research in 2011. He works in the field of advanced driver assistance systems and automated parking.

{\bf Dr. Jens Spehr} received the diploma degree in electrical engineering at TU Braunschweig, Germany, in 2006. He was a research assistant at Institut fuer Robotik und Prozessinformatik at TU Braunschweig, from which he obtained his Ph.D. degree in 2013. He is currently with the Volkswagen Group Research, where he works on scene understanding for intelligent vehicles. His research interests include computational models of vision, machine learning, and artificial intelligence.

{\bf Dr. Thorsten Graf} received his doctoral degree in Computer Science with focus on image processing and artificial intelligence at the University of Bielefeld in 2000. He joined VW in 2001 and worked in many projects on advanced driver assistance systems as member and project leader. Currently, he is the project leader of the environment perception within the department Sensors and Fusion at Volkswagen Group Research.

{\bf Peter Muehlfellner} Peter Muehlfellner obtained his Master’s degree from Halmstad University, Sweden in 2011. From 2011 to 2015, whilst being with Volkswagen Group Research he was also a doctoral student with Halmstad University, working on the EC funded project V-Charge. He expects to defend his thesis in May 2015. His research focused on localization and mapping for automated vehicles using computer vision, with a special interest in the long-term operation of driverless cars. Now he is responsible for occupancy grid mapping and the modeling of unstructured environments at Volkswagen Group Research.


\subsubsection{\ETHZ}

\newHeaderBox{Autonomous Systems Lab}{\bf\ETHZ}

Consistently ranked among the top universities in the world, ETH Zurich is an academic institution with excellent track record of research and teaching since
1885. The university is associated with 21 Nobel Laureates who have studied, taught or conducted research there. The Autonomous Systems Lab (ASL) of ETH
Zurich, is internationally renowned in the field of autonomous robot design and navigation with great experience in design and navigation of wheeled, legged and flying robots operating in different types of environments. With involvement in several National, European and ESA projects together with both academic and industrial partners, the lab currently hosts more than 50 researchers and application engineers. Research achievements in the lab have been commercialized by realizing seven spin-off companies since its foundation.

ASL has been involved in the following European projects:  V-Charge, Noptilus, EUROPA, NIFTi, BACS, COGNIRON, muFly,
Sky-Sailor, SPARC, RCET, Robots@home, URUS, sFly, myCopter, AIRobots, ICARUS, Flourish.

\begin{keypubs}{\ETHZ}
	\item
	M{\"u}lfellner, P.; B{\"u}rki, M.; Bosse, M.; Derendarz, W.; Philippsen, R.; Furgale, P., "Summary Maps for Lifelong Visual Localization", Journal of Field Robotics, March 2015
	\item
	Furgale, P.; Schwesinger, U.; Rufli, M.; and others, "oward automated driving in cities using close-to-market sensors: An overview of the V-Charge Project", IEEE Intelligent Vehicles Symposium, 2013
	\item
	Cieslewski, T.; Lynen, S.; Dymczyk, M.; Magnenat, S.; Siegwart, R., "Map API-Scalable Decentralized Map Building for Robots", 2015
	\item
	Dymczyk, M.; Lynen, S.; Cieslewski, T.; Bosse, M.; Siegwart, R.; Furgale, P., "The Gist of Maps -- Summarizing Experience for Lifelong Localization", Robotics and Automation (ICRA), IEEE International Conference on, 2015
	\item
	Lynen, S.; Bosse, M.; Furgale, P.; Siegwart, R., "Placeless Place-Recognition", 2015
	\end{keypubs}

\newHeaderBox{Principal Investigators}{\bf\ETHZ} {\bf Prof.~Dr.~Roland Siegwart} is a full professor for autonomous systems at ETH Zurich since July 2006 and Vice President of Research and Corporate Relations since January 2010. He received his Diploma in Mechanical Engineering in 1983 and his Doctoral Degree in 1989 from ETH Zurich. He then spent one year as postdoctoral fellow at Stanford University. Back in Switzerland, he worked from 1991 to 1996 part time as R\&D director at MECOS Traxler AG and as lecturer and deputy head at the Institute of Robotics, ETH Zurich. In 1996, he was appointed as professor for autonomous microsystems and robots at the Ecole Polytechnique F\'ed\'erale de Lausanne where he served among others as member of the direction of the School of Engineering (2002-06) and funding chairman of the Space Center EPFL. Roland Siegwart is a board member of the European Network of Robotics (EURON), and served as Vice President for Technical Activities (2004/05) and is currently Distinguished Lecturer (2006/07) of the IEEE Robotics and Automation
Society. Recently he has been appointed as Member of the Swiss Academy of Engineering Sciences and the ``Bewilligungsausschuss Exzellenzinitiative'' of the Deutsche Forschungsgemeinschaft. His research interests are in the design and control of systems operating in complex and dynamic environments. His major goal is to find new ways to deal with uncertainties and enable the design of highly interactive and adaptive systems. Prominent application examples are personal and service robots, planetary exploration robots, autonomous micro-aircrafts and driver assistant systems.


\subsubsection{\IBM}

\newHeaderBox{Department of Cognitive Computing and Computational Sciences}{\bf\IBM}
IBM Research - Zurich, established in 1956, represents the European branch of IBM Research along with its sister lab in Dublin, Ireland. The lab is comprised of scientists representing more than 40 different countries. IBM Research - Zurich is world-renowned for its outstanding scientific achievements - most notably Nobel Prizes in Physics in 1986 and 1987 for the invention of the scanning tunneling microscope and the discovery of high-temperature superconductivity, respectively. The Zurich laboratory is involved in more than 80 joint projects with universities throughout Europe, in research programs established by the European Union and the Swiss government, and in cooperation agreements with research institutes of industrial partners.

This Department of Cognitive Computing and Computational Sciences focuses on next generation cognitive systems and technologies, big data and secure information management, HPC and computational sciences. The synergy of these thrust areas will allow for the integration of data driven discovery and advanced simulations, which is bound to greatly leapfrog the capabilities of conventional IT solutions. With relevance to this call, the group has been involved in the following FP7 projects: NANOSTREAMS, EXA2GREEN, and TEXT.

\begin{keypubs}{\IBM}
  \item R. Cogill, O. Gallay, W. Griggs, C. Lee, Z. Nabi, R. Ordonez, M. Rufli, R. Shorten, T. Tchrakian, R. Verago, F. Wirth and S. Zhuk. Parked Cars as Service Delivery Platform, ICCVE, 2014.
  \item F. Faisal, Y. Iniechen, C. Malossi, P. Staar, C. Bekas and A. Curioni. Massively Parallel and near Linear Time Graph Analytics. SC14, 2014.
  \item C. Malossi, Y. Iniechen,  C. Bekas, A,  Curioni and E. Quintana-Orti. Machine Learning Algorithms for the Performance and Energy-Aware Characterization of Linear Algebra Kernels on Multithreaded Architectures. SC14, 2014.
  \item O. Bhardway, Y. Ineichen, C. Bekas and A. Curioni. Highly Scalable Linear Time Estimation of Spectrograms - A Tool for Very Large Scale Data Analysis. SC13, 2013.
\end{keypubs}

\newHeaderBox{Principal Investigators}{\bf\IBM}

{\bf Dr.~Martin Rufli} is Research Staff Member in the Foundations of Cognitive Computing Group at IBM Research -- Zurich. He received his B.Sc. and M.Sc. degrees in Mechanical Engineering as well as a Ph.D. degree in Robotics from ETH Zurich (Switzerland) in 2006, 2008, and 2012, respectively. He conducted his M.Sc. thesis on Robotic Motion Planning at Carnegie Mellon University (USA), and completed a research stint at Harbin Institute of Technology (P.R. China). From 2012 to 2013 he was a Postdoctoral Research Fellow with the Autonomous Systems Lab at ETH Zurich. He joined IBM as a Research Staff Member in 2013. His research at IBM focuses on the design and  development of a highly scalable cloud-backed localization and mapping framework with applications in IoT and Robotics.


{\bf Dr.~Costas Bekas} is Manager of the Foundations of Cognitive Computing Group within the Department of Cognitive Computing and Computational Sciences at IBM Research -- Zurich . Dr.\ Bekas
received B.\ Eng., Msc and PhD diplomas, all from the Computer Engineering \&
Informatics Department, University of Patras, Greece, in 1998, 2001 and
2003 respectively. In 2003-2005, he worked as a postdoctoral associate with
Professor Yousef Saad at the Computer Science \& Engineering Department,
University of Minnesota, USA. Dr.\ Bekas's research agenda
spans large scale analytics and cognitive systems with an emphasis in graph algorithms/DBs,
numerical and combinatorial algorithms, energy aware and fault tolerant
systems/methods and computational science. Dr.\ Bekas brings more than 10
years of experience in high performance computing and large scale industrial
problems with an emphasis in analytics applications. Dr.\ Bekas was a
recipient of the 2012 Prace Award and the 2013 ACM Gordon Bell prize.


\subsubsection{\CLUJ}

\newHeaderBox{Image Processing and Pattern Recognition Research Centre}{\bf\CLUJ}

Technical University of Cluj-Napoca (UTC) is one of the most prestigious higher education institutions in Romania and dates back to 1920. Study programs in Automation and in Computer Science were started since 1977. Since its birth, the Faculty of Automation and Computer Science has earned an international reputation for its high academic standards and excellence in research. The Computer Science Department approaches a wide range of research topics which addresses fundamental issues in Computer Science and Information Technology like as design and analysis of algorithms, data communications, artificial vision and image processing, distributed systems. The Department is involved in national and international research grants and contracts in cooperation with other universities or industrial partners.

The Image Processing and Pattern Recognition Research Centre (IPPRRC), part of the Computer Science Department is involved in multiple computer vision related research subjects, a significant part of the effort being directed to finding original, robust and fast solutions for environment perception: high accuracy feature-based stereovision, high accuracy dense stereovision, high accuracy dense optical flow, vision based ego-motion estimation using a stereo system, lane detection and tracking, detection and classification of painted road objects, obstacle detection and tracking, obstacle classification, perception and representation of unstructured environments, forward collision detection, dynamic environment perception, high level reasoning on perception and domain knowledge.

IPPRRC has been involved in the EU funded projects R5-COP, PAN-Robots, DRIVE C2X, INSEMTIVES, LarKC, and INTERSAFE-2.

%--"R5-COP -- Reconfigurable ROS-based Resilient Reasoning Robotic Cooperating Systems" , FP7 ARTEMIS project (2014-2017): project aims to provide the means for a fast and flexible adaption of robots to quickly changing environments and conditions to enable a safe and direct human/robot cooperation and interaction at an industrial scale;
%
%--"PAN-Robots -- Plug And Navigate ROBOTS for smart factories", FP7 project (2012-2015): project aims at developing a highly automated logistics system supporting future factories to achieve maximum flexibility, cost and energy efficiency as well as accident free operation;
%
%--"DRIVE C2X - Accelerate cooperative mobility", FP7 project (2011-2014): project aimed at planning, organizing and evaluating field operational tests for cooperative systems using Car-2-X communication
%
%--"INSEMTIVES -- Incentives for semantics", FP7 project (2010-2012): project aimed at producing methodologies, methods and tools that enable the massive creation and feasible management of semantic content in order to facilitate the world-wide uptake of semantic technologies;
%
%--"LarKC -- The Large Knowledge Collider", FP7 project (2010-2011): project aimed to design an integrated pluggable massive distributed incomplete reasoning platform for large-scale semantic computing;
%
%--"INTERSAFE-2 -- Cooperative Intersection Safety", FP7 project (2008-2011): project aimed to develop and demonstrate a cooperative intersection safety system that is able to significantly reduce injury and fatal accidents at intersections.


\begin{keypubs}{\CLUJ}
\item
Radu Danescu, Sergiu Nedevschi, "A Particle-Based Solution for Modeling and Tracking Dynamic Digital Elevation Maps", in IEEE 
Transactions on Intelligent Transportation Systems, vol. 15, no. 3, pp. 1002-1015, 2014.
\item
Marius Drulea, Sergiu Nedevschi, "Motion Estimation Using the Correlation Transform", in IEEE Transactions on Image Processing, vol. 22, no. 8, pp. 3260-3270, 2013.
\item
Sergiu Nedevschi, Voichita Popescu, Radu Danescu, Tiberiu Marita, Florin Oniga, "Accurate Ego-Vehicle Global Localization at 
Intersections Through Alignment of Visual Data With Digital Map", in IEEE 
Transactions on Intelligent Transportation Systems, 
vol. 14, no. 2, pp. 673-687, 2013.
\item
Cosmin Pantilie, Sergiu Nedevschi, "SORT-SGM: Subpixel Optimized Real-Time Semiglobal Matching for Intelligent Vehicles", in 
IEEE Transactions on Vehicular Technology, vol. 61, no. 3, pp. 1032-1042, 2012.
\item
Istvan Haller, Sergiu Nedevschi, "Design of Interpolation Functions for Subpixel-Accuracy Stereo-Vision Systems", in IEEE Transactions on Image Processing, vol. 21, no. 2, pp. 889-898, 2012.

\end{keypubs}

\newHeaderBox{Principal Investigators}{\bf\CLUJ}
\textbf{Prof. Dr. Eng. Sergiu Nedevschi} is a full professor in computer science at the Technical Unversity of Cluj-Napoca since 1998. He founded the Image Processing and Pattern Recognition Research Centre (IPPRRC) at the Technical University of Cluj-Napoca. He was the Head of Computer Science Department between 2000 and 2004, the Dean of Faculty of Automation and Computer Science between 2004 and 2012 and now is vice-rector with scientific research. His research interests include image processing, pattern recognition, computer vision, stereovision, sensorial perception applied in driving assistance and autonomous driving. Prof. Nedevschi is the manager of several national and international projects. He has published multiple papers in refereed conferences and journals in the field of image processing, pattern recognition and sensorial perception applied in driving assistance and autonomous driving.

\textbf{Assoc. Prof. Dr. Eng. Tiberiu Marita} teaches image processing, pattern recognition, and design with microprocessors. His research interests include software and hardware design, object oriented programming, artificial intelligence and computer vision. His computer vision areas of expertise are camera calibration, stereovision, vision based automotive applications and medical imaging. Dr. Marita was team leader in multiple international and national research projects, and has published multiple scientific papers in the related fields.

\textbf{Assoc. Prof. Dr. Eng. Radu Danescu} teaches image processing and design with microprocessors. His research interests include: software and hardware design, object oriented programming, and computer vision. Dr. Danescu's computer vision areas of expertise are: stereovision, road and lane delimiting features extraction, model based and model free probabilistic tracking of structured and unstructured traffic scenes and their components (lanes, vehicles, pedestrians, etc). He was involved in multiple national and international research projects, and has authored multiple scientific papers.

\textbf{Assist. Prof. Dr. Eng. Florin Oniga}'s research interests include computer architecture, image processing and pattern recognition. His computer vision areas of expertise are related to real-time stereovision solutions, real-time feature extraction, and environment perception using Digital Elevation Maps. Florin Oniga was involved in multiple research projects, and has authored multiple scientific papers.

\subsubsection{\PRAGUE}

TODOCVUT

\newHeaderBox{X Y Z Department}{\bf\PRAGUE}
abc

\begin{keypubs}{\PRAGUE}
\item
abc
\item
def
\item
ghi
\item
jkl
\item
mno
\end{keypubs}

\newHeaderBox{Principal Investigators}{\bf\PRAGUE} {\bf Dr.~FirstName LastName} 

\textbf{Dr.~FirstName LastName} is ...

\textbf{Dr.~FirstName LastName} is ...



%\clearpage
\subsection{Third parties involved in the project}
\label{sec:thirdparties}
% Please complete, for each participant, the following table (or simply
% state "No third parties involved", if applicable):
%
%Does the participant plan to subcontract certain tasks  (please note
%that core tasks of the project should not be sub-contracted)
%Y/N
%If yes, please describe and justify the tasks to be subcontracted

%Does the participant envisage that part of its work is performed by
%linked third parties (A third party that is an affiliated entity or has
%a legal link to a participant implying a collaboration not limited to
%the action. (Article 14 of the Model Grant Agreement).)
%Y/N
%If yes, please describe the third party, the link of the participant to
%the third party, and describe and justify the foreseen tasks to be
%performed by the third party

%Does the participant envisage the use of contributions in kind provided
%by third parties (Articles 11 and 12 of the General Model Grant
%Agreement)
%Y/N
%If yes, please describe the third party and their contributions
No third parties involved.

