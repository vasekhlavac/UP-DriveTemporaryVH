% !TEX root = ../proposal.tex

\subsection{Relation to the Work Programme}
\label{sec:relevance}
% Indicate the work programme topic to which your proposal relates, and
% explain how your proposal addresses the specific challenge and scope
% of that topic, as set out in the work programme.



Following Section~\ref{sec:objectives}, \Project{} focuses on a robotic transport application in low-speed urban neighborhood environments where an expedited market take-up appears achievable. It jointly targets at maturing associated base technologies towards higher TRL -- as encouraged by the Roadmap. This procedure, in combination with a strong focus on the transport priority market domain demonstrates the project's strong alignment with the objectives of the call as addressed in detail below.

\begin{center}
\small
  \begin{longtable}[h]{|m{6cm}|m{9.75cm}|}\hline
   {\highlightCell  Scope of the call H2020 ICT-24-2015: Robotics} & {\highlightCell Objectives of \Project{}}\\ \hline
  \endfirsthead
  \hline
    {\highlightCell Challenge and Scope of the call H2020 ICT-24-2015: Robotics} & {\highlightCell Objectives of \Project{}}\\ \hline
    \endhead
       \textit{Collaborative projects will cover multi-disciplinary R\&D and innovation activities like technology transfer via use-cases and industry-academia cross fertilisation mechanisms.} & The \Project{} consortium consists of a balanced team of universities, and industrial partners, all of which are internationally leading in their respective domains. At least one academic and one industrial partner are involved in each of the key work packages. This ensures not only good overall system integration but also facilitates knowledge transfer: the involvement of industrial partners strengthens the end-user perspective of the project, whilst the academic partners offer unique expertise in the respective fields. The research work performed in the universities and research labs will be implemented and demonstrated on modern vehicles and commercially available server infrastructure. More importantly: the sensor setup of the test vehicles will be based on production or pre-development automotive sensors. That -- combined with the adoption of common industry standards for interfaces -- maximizes the chances of incorporating project results into future market products/exploitation.\\ \hline
    \textit{RTD to advance abilities and key technologies relevant for industrial and service robotics. In terms of market domains, the priorities are: healthcare, consumer, transport.} & \Project{} addresses the transport market domain with a focus on low-speed urban outdoor environments via an on-street valet parking application. As will be described in Section 1.3. in detail, this application has been selected due to its representativeness of the general challenges of automated urban driving. The advancement in base technologies necessary for its success -- multi-modal perception, lifelong localization and mapping and scene understanding -- will thus simultaneously advance large parts of applications in the priority market domains and beyond.\\ \hline
      \textit{The primary goal is to significantly improve [...] adaptability, cognitive ability, configurability, decisional autonomy, dependability, flexibility, interaction capability, manipulation ability, motion capability, perception ability.} & \Project{} aims to advance the key technologies necessary for autonomous operation in dynamic urban environments. One goal is to develop novel perception and scene understanding techniques. Multi agent collaboration -- both online, as well as based on aggregating past data -- will be exploited. This will be accompanied by development of techniques for accurate, lifelong localization, which in turn is a key enabler technology for sharing perception data among multiple agents. In order to close the sense-plan-act loop strategies for autonomous decision making, as well as motion planning techniques for dynamic environments will be investigated. In order to improve on the dependability, attention will be given to assessing the data integrity and resolving situations in which this integrity -- due to external or internal factors -- is reduced. \\ \hline
  \textit{To prove the exploitation potential of the results the project outcome is to be shown in market domain-relevant demonstrations proving an increased TRL.} & The \Project{} project will operate on real world datasets from the very beginning. Initial data will be acquired from test platforms developed in previous projects (e.g. V-Charge). Tests will be performed within real world environments of different complexities and during different weather conditions and times of day/night. The integrated outcome of the project will be demonstrated in closed or open neighborhood areas at regular intervals throughout the project. This will be accompanied by demonstrations of each of the technology improvements mentioned in Section \ref{sec:trl}. The project expects to reach TRL 6 for the integrated system.\\ \hline
\end{longtable}
\end{center}

