% !TEX root = ../proposal.tex

\subsubsection{Regulatory barriers and obstacles}
The legal framework in Europe does not presently permit unaccompanied automated driving in public areas. And while there is substantial work underway to change this in the future, this is a longer-term proposal. Hence, a human operator must be present to ensure safety and accountability during all \Project{} test and demonstration drives. While this does not pose an immediate concern for demonstration purposes, it represents significant barrier for making general automated systems available on the market. 

\Project{} aims to overcome this difficulty by proposing and demonstrating a targeted application serving \emph{specific, important citizen needs} in a restricted environment, while supporting the further development of broadly needed base technological competences. This approach in turn is expected to rapidly enable a scale-out of infrastructure composed of (partially automated) cars that will facilitate the wide-spread real-time perception required as a basis for safe transportation -- thereby paving the way for accelerated changes to the legal framework from the technological side.


\subsubsection{Impact on Applications}
\label{sec:apps}
The technologies developed within \Project{} aim at a collaborative automated robotic system able to operate in dynamic urban environments. In addition to automated transportation such systems have a large number of possible applications in widely-varying areas, including high resolution 2D and 3D mapping, environmental monitoring and knowledge representation\,/\,reasoning for traffic management, surveillance, inspection (industrial and in agriculture), freight delivery, and construction. Furthermore, besides these active applications, such systems could be used as formidable distributed mobile computation, storage and communication infrastructure.

%In the following, we highlight a few important use cases where the \Project{} technologies could be directly applied.
%\begin{denseItemize}
%\item \textbf{High-resolution mapping:} The maps acquired and managed within \Project{} could be used for a multitude of applications besides robot localization. Examples include augmented reality, geo-services, and others.
%
%\item \textbf{Environmental monitoring:} Up-to-date quantitative information on the environment will become increasingly important as we address
%the problem of climate change. The same strategies for lifelong mapping developed within \Project{} could be used to acquire and maintain non-visual aspects of the environment, such as pollutant concentrations.
%
%\item \todo{Agriculture: }
%\item \todo{Disaster management: }
%\end{denseItemize}

%The Commercial Space Sector: the lifelong mapping and scene interpretation aspects developed within \Project{} could power the future commercial space sector such as exploration.
%
%Commercial Construction: A team of mobile robots could be used to perform daily survey of construction
%sites. The mapping and scene interpretations tools developed within the project could be adapted to enable management to track progress and
%identify problem areas within construction early.
%
%Industrial Inspection: A similar argument applies to industrial inspection. A team of robots enables
%large-scale survey of industrial sites, the multi-spectral maps allow inspection for geometry and heat emission,
%and continuous survey strategy will allow problems to be identified early, potentially stopping problems
%before they happen and increasing safety for workers.


\subsubsection{Why a European Approach is Needed}
\label{sec:european_approach}
The \Project{} consortium brings together specialist researchers, technology, and leading solution providers from different European countries, each providing expert skills to the planned activities. While some of the know-how in this project would be available on a national level, the European level is expected to have much greater strategic impact.
\begin{denseItemize}
\item The competences brought in by each partner are summarized in Section~\ref{sec:consortium}. Together, they fully cover the spectrum of know-how needed to carry out the proposed project. Such a unique, complete and gap-less combination of multi-disciplinary expertise, particularly at a globally leading level, is much more difficult (if not impossible) to find and commit in any single member state.
\item The exploitation and reuse of results by the industrial project partners, but also by other
  academic (i.e., via startups) and industrial stakeholders (via d) that benefit from the generated know-how and functionality would be much more limited in
  scope and much more difficult to achieve.
\item The critical mass and visibility created by a European level activity are essential to generating reuse of results and, in particular, to facilitate the introduction of advanced solutions in professional environments.
\end{denseItemize}
All consortium partners are involved in and have have had experience with European and national projects and are well connected in national and
international networks. The visibility as a European project will maximize the chance of re-use outside of the consortium and, hopefully, in turn extension of project results by others.


%\subsection{Dissemination, Exploitation, and Intellectual Property Handling}
\subsection{Measures to maximize impact}
\label{sec:impact-measures}
\subsubsection{Dissemination and exploitation of results}
%   - Provide a draft ��plan for the dissemination and exploitation of
%     the project's results�� (unless the work programme topic explicitly
%     states that such a plan is not required). For innovation actions
%     describe a credible path to deliver the innovations to the market.
%     The plan, which should be proportionate to the scale of the project,
%     should contain measures to be implemented both during and after the
%     project.  
%     
%     Dissemination and exploitation measures should address the full
%     range of potential users and uses including research, commercial,
%     investment, social, environmental, policy making, setting
%     standards, skills and educational training. 
%
%     The approach to innovation should be as comprehensive as possible,
%     and must be tailored to the specific technical, market and
%     organisational issues to be addressed.  
%  - Explain how the proposed measures will help to achieve the expected
%    impact of the project. Include a business plan where relevant.
%  - Where relevant, include information on how the participants will
%    manage the research data generated and/or collected during the
%    project, in particular addressing the following issues:
%    * What types of data will the project generate/collect?
%    * What standards will be used?
%    * How will this data be exploited and/or shared/made accessible for
%      verification and re-use? If data cannot be made available, explain
%      why.
%    * How will this data be curated and preserved?
%
%    You will need an appropriate consortium agreement to manage
%    (amongst other things) the ownership and access to key knowledge
%    (IPR, data etc.). Where relevant, these will allow you,
%    collectively and individually, to pursue market opportunities
%    arising from the project's results. 
%    
%    The appropriate structure of the consortium to support exploitation
%    is addressed in section 3.3.
%  
%  - Outline the strategy for knowledge management and protection.
%    Include measures to provide open access (free on-line access, such
%    as the ‘green’ or ‘gold’ model) to peer-reviewed scientific
%    publications which might result from the project. 
%  
%    Open access publishing� (also called 'gold' open access) means that
%    an article is immediately provided in open access mode by the
%    scientific publisher. The associated costs are usually shifted away
%    from readers, and instead (for example) to the university or
%    research institute to which the researcher is affiliated, or to the
%    funding agency supporting the research. 
%    
%    Self-archiving (also called 'green' open access) means that the
%    published article or the final peer-reviewed manuscript is archived
%    by the researcher - or a representative - in an online repository
%    before, after or alongside its publication. Access to this article
%    is often - but not necessarily - delayed (‘embargo period’),  as
%    some scientific publishers may wish to recoup their investment by
%    selling subscriptions and charging pay-per-download/view fees
%    during an exclusivity period.

\subsubsection*{Dissemination Activities}
\label{sec:diss}

Dissemination will be an important task for the \Project{} project. Dissemination tools will be installed (for example, web-site, software releases and repositories, \etc) and an active contribution to conferences and publications is a central aspect of \Project{}. A number of workshops and similar events could be proposed for dissemination both inside and outside the consortium.

% Dissemination will take place on two levels:
% \begin{denseItemize}
% \item {\bf Local dissemination} represents the internal dissemination
%   of knowledge with each partners site as well as in the consortium
%   itself.  Intra-consortium dissemination concentrates on the
%   technologies and tools developed within the project, the
%   interrelation of project-specific work with other projects performed
%   within the consortium and is mainly performed through internal
%   presentations as well as internal meetings.

%   To strengthen the exchange of information between members of the
%   consortium, {\bf partner exchanges} will be carried out.  The
%   possibilities for members of the participants to temporarily work at
%   the partners' site typically boosts the exchange of knowledge and
%   supports new, fruitful collaborations.

% \item {\bf General dissemination} is addressed to the research
%   community, European industry and the wider public that will profit
%   from the project. Its aim is to inform and stimulate interest. Wider
%   dissemination will include participation in scientific conferences
%   and workshops, commercial exhibitions and fairs, science days for
%   the public, clustering activities of the FP7-ICT programme and also
%   press releases.  Effective dissemination and exploitation will be
%   ensured through the set-up of a project web-page with \Project's
%   results, the presentation of \Project{} achievements at
%   international conferences and in journals, as well as the
%   organization of public workshops.
% \end{denseItemize}

The \textbf{key tools for dissemination} are:
\begin{denseItemize}
\item {\bf Dissemination via Public Events.} All partners are typically present at information days for the general public. This includes exhibitions and special events at museums. In addition, the university partners often participate in ``science days'' and similar events to increase the awareness of technology in the public. We will exploit these forums for disseminating the results of \Project{}. For the industrial dissemination, presence in fairs and exhibitions as well as the distribution of information material about the \Project{} technologies are the key activities. In addition to that, the consortium members will use their connections and collaborations to current past industrial partners to disseminate the results.  

\item {\bf Internet Presence \& Social Media:} The consortium will set up a web-site as a dissemination tool. The web-site will be maintained during the lifetime of the project. It will contain up-to-date news about the progress of the project, information about the presence of the project in conferences, fairs, exhibitions, \etc., a YouTube channel publicizing the aims and results of the project, and the possibility to download of scientific papers and related documents. For the last point, we will follow the `green' open access strategy by self-archiving preprints of scientific papers. The website will be linked with social media and news feed platforms including {\bf Twitter and Facebook}.

\item {\bf Printed Materials:} We will also produce printed materials for the project dissemination within the European community. These materials include leaflets with general information about the project objectives, brochures with information about the final system and project posters for conferences, exhibitions, \etc.

\item {\bf Scientific Publications:} All technical details about the methods and technology developed in \Project{} will be public. The documents will be in the form of scientific papers that will be published at peer-reviewed conferences and journals. All material will be made available at the project web-site.

\item {\bf Dataset Publication:} Some portion of the data that we produce in \Project{} may be made public to benefit other researchers and drive innovation. For example, we may develop a benchmarking dataset for lifelong mapping and scene interpretation in urban neighborhoods. \ETHZ{} already has the necessary infrastructure to archive and host this data. We will consider applying the same strategy to other data produced in the project while respecting the specific policies of the consortium members -- particualy the industrial partners. This will be detailed in the data management plan (see \WPInnovation)

\item {\bf Software Releases:} The academic members of the consortium are actively involved in releasing open-source software based on their scientific research. This strategy will be maintained for the \Project{} project.
\end{denseItemize}

The \textbf{dissemination towards ICT and Robotics} focuses on the scientific community and industry.  The main action line within the dissemination activities is that of \textbf{scientific and technological dissemination}. All the partners will actively contribute to the project's dissemination activities by presenting the results of the project in well-known and widely read international scientific journals and also by presentations in international scientific conferences,
workshops and exhibitions.  A major aspect of {\bf scientific dissemination} regards technical publications and presentations in established journals and conferences in the area. The key target conferences include:

\begin{megaDenseItemize}
  \item IEEE European Conference on Computer Vision
  \item IEEE International Conference on Computer Vision
  \item IEEE International Conference on Computer Vision and Pattern Recognition 
  \item IEEE International Conference on Robotics and Automation
  \item IEEE/RSJ International Conference on Intelligent Robots and Systems
  \item International Joint Conference on Artificial Intelligence
  \item Robotics: Science and Systems
  \item International Symposium on Robotic Research
  \item International Symposium on Experimental Robotics
  \item IEEE Intelligent Vehicles Symposium
  \item Intelligent Transport Systems World Congress
  \item IEEE Intelligent Transport Systems Conference
  \item IEEE Transactions on Intelligent Transportation Systems
  \item IEEE Transactions on Robotics
  \item IEEE Intelligent Transportation Systems Magazine
\end{megaDenseItemize}


\subsubsection*{Exploitation Activities}
\label{sec:expl}
The objective of these activities, carried out within \WPInnovation{}, aim at the future commercial exploitation of the developed system, parts thereof and underlying technology -- thus paving the way for a successful massive product deployment. It consists of the following general activities and specific exploitation plans:

\textbf{Identification of the exploitable results:}  Identify the components, protocols, tools and individual results that can be exploited individually and applied to project application scenarios as well as those directly related to the project developments (as outlined in Section~\ref{sec:apps}). This also includes the identification of project contributions to technical standards.

\textbf{Exploitation of project results:}  Project results will be employed to the greatest possible advantage for the companies and research institutions working in the project to their own products and research activities according to the principles established in the consortium agreement. The results may be integrated, adapted or improved to fit new scenarios. Below we list partner-specific exploitation intentions.

TODOCVUT own exploitation plans?

Volkswagen Group Research is continuously working together with different brands of the Volkswagen Group with the goal of accelerating the arrival of innovations to the market. This cooperation has many different forms, including:
\begin{denseItemize}
  \item Transfer of concepts, know-how, prototype source code generated in the research projects into the development and pre-development departments
	\item Evaluations of application of the prototype research systems / solutions to use-cases of interest for the development and pre-development departments
	\item Evaluation of the maturity levels of the technologies, identification of weaknesses that need to be tackled and eliminated before market introduction
	\item Discussions of non-technical or secondary constraints for market introductions of systems, or subsystems: costs, size, certification issues, etc... This helps the Group Research to adopt realistic premisses in research projects
	\item Demonstrations of new technologies or prototype systems to the decision-makers within the company
\end{denseItemize}
The role of the last point in market introduction of the innovations can hardly be overestimated. That is why \Project is designed such to deliver a fully integrated system able to perform convincing demonstrations.
Volkswagen Group Research expects that especially in the fields of localisation, local perception and navigation \Project will provide results of very high relevance to the (pre-)development departments of the brands VW and Audi. This is even more probable, as \Project is based on close-to-market sensors.

At \ETHZ, the efforts into developing a lifelong localization and mapping framework will be carried out in collaboration with and directly exploited by the spin-off called Zurich-Eye. The goal of Zurich-Eye is to take results from cutting-edge research conducted by both \ETHZ and the University of Zurich on visual-inertial navigation, lifelong mapping and vision-based localization and transfer them into market-ready industrial applications. In addition to autonomous driving, a wide range of application domains, including pedestrian indoor navigation, architecture, hospital and warehouse logistics, repeatable inspection, guidance for museums and trade fairs, survey on construction sites as well as navigation aids for the visually impaired, will benefit from these results. 

Part of \IBM's exploitation strategy is to open new doors to its strategic focus areas cloud, big data and analytics, mobile, social and security (CAMSS). While traditionally these areas have been served from a  backend, \Project{} forms an innovative hybrid between offline analytics and real-time Robotics -- characteristic for many future applications. In this regard, the lifelong map management and scene interpretation aspects of \Project{} may form an effective extension to IBM's existing cognitive capabilities grouped under the Watson umbrella.

\CLUJ has since 1998 a research entity called "Image Processing and Pattern Recognition Research Centre" (IPPRRC) working in the field of vision based perception. The research results of this group were obtained through projects funded by national and European agencies but especially by companies involved in automotive industry like Volkswagen and Bosch. 
Basic components for stereovision based perception were developed including stereo-reconstruction engines, optical flow engines and visual odometry based ego-motion estimation solutions. On the top of these components, perception applications for highway, urban and intersection environments were developed. In parallel with scientific results, a significant increase in human resource quality and quantity was obtained. 
In the framework of this project, starting from the paradigm of the spatio-temporal and appearance based low-level representation, \CLUJ intends to increase the robustness and the performances of the perception solutions for reaching the technology demonstration level. 
Based on these achievements UTC investigates better possibilities to exploit the scientific and human resources. A spin-off based solution is an option.




\subsubsection{Contributions to Standards}
The primary goal of \Project is to push the state of the art for the automated driving technologies. Nonetheless, the consortium partners will aim to adhere as much as possible to existing standards as well as such under formation, as relevant to the project. The members of the consortium further agreed to contribute to standardization activities in robotics and transportation and participate in or join different task groups for establishing standards. Below we list ongoing or planned activities with partner involvement:

\begin{denseItemize}
\item Contribution and adherence to the IEEE/RAS Standard for Robot Map Data Representation for Navigation.\footnote{\url{https://ieee-sa.centraldesktop.com/1873workinggrouppublic/}}
\item Contribution to various aspects of the connected Intelligent Transportation System (C-ITS) Platform.\footnote{\url{http://ec.europa.eu/transport/themes/its/news/c-its-deployment-platform_en.htm}} IBM actively participates in various aspects of the platform.
\item Contribution to Navigation Data Standard Association\footnote{\url{http://www.nds-association.org/}}, a body focused on standardization of the navigation maps used in the automotive industry. \VW is part of NDS Association and will regularly provide input to the standardization group, to ensure that future navigation maps will support the applications developed within \Project
\item \VW is taking part in the ''Roundtable for Automated Driving''\footnote{\url{https://www.vda.de/en/topics/innovation-and-technology/automated-driving.html}} established by the German Ministry of Transport with the support of the ''Verband der Automobilindustrie''. This body aims at preparing changes of the regulatory system to allow the gradual introduction of automated driving in Germany.
\end{denseItemize}



\subsubsection{Intellectual Property Handling}
\label{sec:iprhandling}
For the success of the project it is essential that all project partners agree on explicit rules concerning Intellectual Property IP ownership, access rights to any Background and Foreground IP for the execution of the project and the protection of intellectual property rights (IPRs) and confidential information before the project starts. Such issues will be addressed in detail within the Consortium Agreement between all project partners. The partners have already discussed and agreed on the following guidelines during the preparation of the proposal:\\
\textbf{Access Rights to Background and Foreground IP during the project:} In order to ensure a smooth execution of the project, the project partners agree to grant each other royalty-free access rights to their Background and Foreground IP for the execution of the project. Any details concerning the access rights to Background and Foreground IP for the duration of the project will be defined in the Consortium Agreement.\\
\textbf{IP Ownership:} Foreground IP shall be owned by the project partner carrying out the work leading to such Foreground IP. If any Foreground IP is created jointly by at least two project partners and it is not possible to distinguish between the contribution of each of the project partners, such work will be jointly owned by the contributing project partners. The same shall apply if, in the course of carrying out work on the project, an invention is made having two or more contributing parties contributing to it, and it is not possible to separate the individual contributions. Any such joint inventions and all related patent applications and patents shall be jointly owned by the contributing parties. Any details concerning the exposure to jointly owned Foreground IP, joint inventions and joint patent applications will be addressed in the Consortium Agreement.



\subsubsection{Communication activities}
\label{sec:communicationact}
%  - Describe the proposed communication measures for promoting the
%    project and its findings during the period of the grant. Measures
%    should be proportionate to the scale of the project, with clear
%    objectives.  They should be tailored to the needs of various
%    audiences, including groups beyond the project's own community.
%    Where relevant, include measures for public/societal engagement on
%    issues related to the project. 

To promote the \Project{} project, the scientific results, and the benefit to players in the Automotive and Transportation domains, we must {\em demonstrate} our advances in a real-world urban environment, and {\em communicate} the results to the public. We propose to do this with a {\bf coordinated communication strategy} to publish our results widely and {\bf engage the public}.

This will involve two events and ongoing media activities:
\begin{denseItemize}
  \item {\bf Press Video M27:} We will produce a press video of the project based on the results of our integration and testing week in M24 (See Section~\ref{sec:workoverview} for details about our spiral development model and testing week schedule). The press video will describe the project and its goals in plain language and highlight some of the key scientific and integration targets achieved at the midpoint of the project. The video will be released in coordination with the communications departments of all partners along with a press release and press photos. This is a strategy that some members of the consortium used for the V-Charge project ({\footnotesize \url{http://www.v-charge.eu/?p=614}}) and we see it as a key tool to communicate the aims and achievements of the project.
  \item {\bf Public Demonstration M48:} We will arrange a public demonstration of the results of the project in M48. The event will be organized in coordination with the communications departments of all partners. The demonstration will take place in a real urban setting within a 30km/h zone. The event will showcase the aims of the project, our scientific achievements, and the market potential of the developed technology. \ETHZ{} had excellent success in the past with the public demonstration of the EUROPA project ({\footnotesize \url{https://www.youtube.com/watch?v=A9A29wpkTaU}}) and we hope to have even greater impact with \Project{}.
\item {\bf Twitter and Website:} The consortium will use Twitter and other social media platforms in combination with the project website to stimulate interactions with user and to promote results. 
\end{denseItemize}
By having a clear communication strategy before the project starts, we will maximize the communication impact of the \Project{} project.

