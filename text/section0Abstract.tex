% !TEX root = ../proposal.tex

\section*{Abstract}
%Automotive industry is showing a clear trend toward the fully automation of driving task. Although the time line for this introduction is still not clear due to legal limitations, there is a tangible evidence of an acceleration in this direction since off-the shelf cars offers more and more systems that automates simple tasks and assist the driver in limited situations, such as parking maneuvers in the nearby of a parking space, adaptive cruise control, and lane following in extra-urban and highways.
%The consortium has identified a set of challenging target situations for an integrated system capable of managing automated driving in urban areas and at low speeds. These situations are not covered by other project presented in  literature. The system can be seen as a parking valet for local areas where the car occupants are left at destination and the car search for a free parking space in the nearby. Creating such a system involves orchestrating several capabilities exploited on-board, and off-board, the vehicle; these includes: local perception, driver behavior intentions, localization and mapping, such as  such as long term mapping and an high level understanding of the scene. The off-board processing will be performed on a cloud infrastructure connected to the vehicle. The demonstration of the system will require multiple prototypes set-up with actuation systems, a mix of perception technologies, adequate processing power, and communication devices.
%Being the target scenarios, local urban areas, much more challenging for this kind of systems compared to highways or parking lots, the results of this project will bring important technological advancement toward the integration and introduction of autonomous driving.

Automation of individual transport systems is considered an up-and-coming prospect with the potential of greatly mitigating many of the challenges associated with intensified urbanization, while at the same time offering additional benefits for the citizens and drastically increasing overall street safety. However, due to the lack of maturity of involved key technologies and persisting legal limitations, full automation of on-road driving remains a longer-term vision, particularly in urban environments. 
The goal and ambition of UP-Drive is to address these technological challenges through the development of an automated valet parking service for city environments, aimed at relieving a car driver from the burden of finding a parking space in city centers. Instead, the fully automated car navigates on its own through urban neighborhoods, finds a parking space and returns on-demand.
Creating such a system requires mastering all key technologies essential to automated urban driving beyond the current state-of-the-art: complete round-view perception of the vehicle environment, robust lifelong localization and mapping, sophisticated understanding of complex scenes as well as aggregation and integration of long-term semantic data over a cloud-based infrastructure. With this, we ensure that the research and development carried out in this project will directly be applicable to other urban driving use-cases such as driver assistance and safety systems on the one hand, and on the other hand to the transportation for elderly and citizens with handicaps, last-mile delivery of goods - and ultimately fully automated urban driving in general.
The consortium will continuously integrate the research and development from all partners into a fully functional vehicle platform and will showcase the end-product in its full extent to the general public.

