% !TEX root = ../proposal.tex

%The advent of autonomous mobile robots into mainstream applications has been heralded for a while now. And indeed, while mostly restricted to operation in laboratory environments at low TRL or in fenced-off high-precision industrial settings for large parts of the 20th century,

In an ongoing massive process towards urbanization, 70\% of the global population is expected to be living in urban or suburban areas by 2050. Along this rapid development, the traffic infrastructure in many urban hubs has become strained despite significant public and private investment into transportation systems and the renewing and extension of road infrastructure. In many cities, severe persistent traffic congestion are a daily occurrence with associated social as well as ecological \& economical challenges, the overall costs of which have been estimated in excess of \euro100 billion per year~\footnote{T.A.T. Institute, Annual Urban Mobility Report. 2012.}.

%not enough parking space

%it is so bad with cars, yet, need for individual mobility remains high

Automated transportation in combination with novel transportation concepts and associated services (both for citizens as well as for goods), is envisioned to eventually greatly alleviate many of these challenges and provide additional benefits.
%Consider for instance the following features and resulting benefits:
%\begin{denseItemize}
%	\item better coordination between vehicles: traffic becomes more efficient 
%	\item removal of human error: increase in safety
%	\item driverless valet parking: no more time wasted on searching for parking spots
%	\item driverless rearrangement of parked cars: more efficient usage of resources such as parking spots and charging stations for electric vehicles
%	\item driverless pick-up at the doorstep: car-sharing becomes more attractive, full individual mobility becomes more affordable
%	\item virtual chauffeur: mobility for the elderly or handicapped, last mile delivery, etc...
%\end{denseItemize}
Consider for instance that via better coordination of vehicles traffic would become more efficient; via removal of human error, safety for all citizens, not only drivers, would be increased; via a virtual chauffeur and pick-up at the doorstep service, car-sharing would becomes more attractive, full individual mobility would become more affordable, mobility for the elderly or citizens with handicaps would be drastically improved, and the delivery of goods on the last mile could be effectively and innovatively approached.

While these applications presently remain visions, over the past decade substantial and impressive progress has been attained in vehicle automation with robotic trials and public funding having engendered an ecosystem from which academia and industry are embarking onto a challenging road map towards fully automated transportation. Yet, due to various technological, standardization and legal hurdles persisting, complete automation for generic on-road driving scenarios appears to remain a longer-term vision -- as pointed out by the Multi-Annual Roadmap. \emph{On the path} towards full automation, however, the \Project consortium considers lack of mature technology in several core aspects to be the main issue presently -- and this particularly holds for urban environments. %It is quite symptomatic that even most of the Advanced Driver Assistance Systems available on the market today originate from -- and address highway-like scenarios. So far, urban areas have been explicitly addressed only by the parking assistance systems. 

%The reason for this discrepancy is the complexity of urban environments and more importantly -- urban traffic. The specific challenges are:
%
%\begin{denseItemize}
%	\item road geometry is more complex than in the highway case: sharper curves, crossroads, traffic isles
%	\item yet lane boundaries might be marked by many different types of delimiters: markings, curbs, change in road surface material or such marking might be completely missing (consider middle of a crossroads)
%	\item more different types of road users are present and need to be detected and taken into account: pedestrians, cyclists, motorcyclists, but also pedestrians with baby-strollers, etc...
%	\item traffic participants are generally moving in all directions (not only longitudinally) and also often perform direction changes (turns, etc...)
%	\item scene is cluttered with many objects (parked cars, lamp posts), which increase the potential of occlusions or misinterpretations
%	\item there are many complex inter-dependencies between traffic participants \todo{good example} 
%\end{denseItemize}

%The reason for this discrepancy is the complexity of urban environments and more importantly -- urban traffic, including more complex road geometry, varying lane boundaries, more diverse road user types, and significantly more cluttered scenes.

%\footnote{\url{http://www.eu-robotics.net/cms/upload//Multi-Annual_Roadmap2020_ICT-24_Rev_B_full.pdf}} and this call.

%In light of this reality, the Roadmap encourages a two-pronged approach to overcome the persisting difficulties by (ii) focusing on specific high-value market domains where an expedited market take-up appears achievable, and (i) maturing associated base technologies towards higher TRL. The \Project{} consortium fully shares and subscribes to this strategy.


%\fbox{
%\begin{minipage}{0.99\linewidth}
%The stated goal of \Project{} is to disrupt and hence accelerate this process by (i) identifying targeted, rather than generic, high-value applications serving particular citizen groups (circumventing legal hurdles), (ii) promoting and adopting an innovative perspective on existing instrumented vehicle infrastructure (enabling an accelerated perception scale-out), and (iii) developing step advancements in key base-technological areas (lifelong localization and mapping, scene understanding) to support the identified applications. Via this mechanism the project aims to achieve a mutually reinforcing cycle of research, development and deployment of increasingly automated services and mobility solutions in urban areas -- thereby paving the technological and infrastructure way for future legislation.
%\end{minipage}
%}


%The objective of this project is to develop a smart car system that allows for autonomous driving in designated
%areas (e.g. valet parking, park and ride) and can offer advanced driver support in urban environments. The final
%goal in four years is the demonstration and implementation of a fully operational future car system including
%autonomous local transportation, valet parking and battery charging on the campus of ETH Zurich and TU
%Braunschweig.
%The envisioned key contribution is the development safe and fully autonomous driving in city-like environments
%using only low-cost GPS, camera images, ultrasonic sensors and radar.
%Within the proposed project, the focus will therefore be set on the following main topics:
%• Development of machine vision systems based upon close-to-market sensor systems (such as stereo vision,
%radar, ultrasonic etc.) as well as the integration and fusion of each sensors data into a detailed world model
%describing static and dynamic world contents by means of online mapping and obstacle detection and tracking.
%• Computer-base situation assessment within the world model as well as describing dependencies and
%interactions between separate model components (e.g. separate dynamic objects). For this purpose, the
%integration of market-ready map-material (i.e. originating from navigation systems) as well as the use of
%vehicle-to-infrastructure communication shall be explored.
%• Precise low-cost localization in urban environments through the integration of standard satellite-based
%technologies with visual map-matching approaches combining both the onboard-perception system and available
%map material.
%• Highly adaptive global and local planning considering dynamic obstacles (cars, pedestrians) and their potential
%trajectory.


